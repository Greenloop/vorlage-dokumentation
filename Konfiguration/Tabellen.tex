\usepackage{booktabs} % für modernen Tabellenlook

% Für die Tabellen gibt es extra custom Commands, um bei Bedarf schnell alle Tabellen einfach umzustylen

% Tabellen styling
\newcommand{\tH}{\emph} % Text Highlight Header z.B. \emph, \textbf, \textit
\newcommand{\tF}{\emph} % Text Highlighting im Footer

% Tabellenstrukturierung
\newcommand{\tableHead}{\rowcolor{cTableHeading}\toprule} % Für die erste Tabellenspalte
\newcommand{\tableBody}{\midrule} % um den Tabellenkörper zu "initialisieren"
\newcommand{\tableBodyLine}{\midrule} % um eine horizontale Linie innerhalb der Tabelle anzuzeigen
\newcommand{\tableSub}[1]{\rowcolor{cTableHeading}\tableBodyLine#1\tableBodyLine} % für Unterbereiche innerhalb der Tabelle
\newcommand{\tableFoot}{\bottomrule\hline} % Um den Hauptteil der Tabelle abzuschließen


% fügt Tabellen aus einer TEX-Datei ein
\newcommand{\tabelle}[3] % Parameter: caption, label, file
{\begin{table}[htbp]
\centering
\singlespacing
\input{Tabellen/#3}
\caption{#1}
\label{#2}
\end{table}}

\newcommand{\tabelleAnhang}[1] % Parameter: file
{\begin{center}
\singlespacing
\input{Tabellen/#1}
\end{center}}