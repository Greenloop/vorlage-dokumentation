\subsection{Benutzerdokumentation}
\label{app:BenutzerDoku}
Ausschnitt aus der Benutzerdokumentation:

\begin{table}[htb]
\begin{tabularx}{\textwidth}{cXX}
\rowcolor{cTableHeading}\bf{Symbol} & \bf{Bedeutung global} & \bf{Bedeutung einzeln} \\
\includegraphicstotab[]{weather-clear.png} & Alle Module weisen den gleichen Stand auf. & Das Modul ist auf dem gleichen Stand wie das Modul auf der vorherigen Umgebung. \\
\rowcolor{cTableOdd}\includegraphicstotab[]{weather-clear-night.png} & Es existieren keine Module (fachlich nicht möglich). & Weder auf der aktuellen noch auf der vorherigen Umgebung sind Module angelegt. Es kann also auch nichts übertragen werden. \\
\includegraphicstotab[]{weather-few-clouds-night.png} & Ein Modul muss durch das Übertragen von der vorherigen Umgebung erstellt werden. & Das Modul der vorherigen Umgebung kann übertragen werden, auf dieser Umgebung ist noch kein Modul vorhanden. \\
\rowcolor{cTableOdd}\includegraphicstotab[]{weather-few-clouds.png} & Auf einer vorherigen Umgebung gibt es ein Modul, welches übertragen werden kann, um das nächste zu aktualisieren. & Das Modul der vorherigen Umgebung kann übertragen werden um dieses zu aktualisieren. \\
\includegraphicstotab[]{weather-storm.png} & Ein Modul auf einer Umgebung wurde entgegen des Entwicklungsprozesses gespeichert. & Das aktuelle Modul ist neuer als das Modul auf der vorherigen Umgebung oder die vorherige Umgebung wurde übersprungen. \\
\end{tabularx}
\end{table}

