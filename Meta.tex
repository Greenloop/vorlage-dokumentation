% Hinweis: der Titel muss zum Inhalt des Projekts passen und den zentralen Inhalt des Projekts deutlich herausstellen
\newcommand{\titel}{Beispiel Projekttitel}
\newcommand{\untertitel}{Beispiel Projektuntertietel}
\newcommand{\kompletterTitel}{\titel{} -- \untertitel}

\newcommand{\pruefungstermin}{Sommer 1000}
\newcommand{\ort}{Musterhausen} % Ort der Abgabe
\newcommand{\abgabetermin}{01.01.1000}
\newcommand{\ausbildungsberuf}{Fachinformatiker für Anwendungsentwicklung}
\newcommand{\betreff}{Dokumentation zur betrieblichen Projektarbeit}

\newcommand{\autor}{Max Mustermann}
\newcommand{\autorStrasse}{Teststraße 1}
\newcommand{\autorOrt}{12345 Berlin}

\newcommand{\logo}{logoBetrieb.png} % pfad zum Logo des Betriebs unterstützt je nach compiler einige typen z.B. jpg, png, pdf, eps
\newcommand{\betriebName}{Musterfirma} % benutzt eine andere Art von text für diesen Teil
\newcommand{\betriebSubName}{IT GmbH}
\newcommand{\betriebStrasse}{Musterstraße 123}
\newcommand{\betriebOrt}{12345 Exmaplehausen}

